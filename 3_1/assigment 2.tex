%%% Template originaly created by Karol Kozioł (mail@karol-koziol.net) and modified for ShareLaTeX use

\documentclass[a4paper,11pt]{article}

\usepackage[T1]{fontenc}
\usepackage[utf8]{inputenc}
\usepackage{graphicx}
\usepackage{xcolor}

\renewcommand\familydefault{\sfdefault}
\usepackage{tgheros}
% \usepackage[defaultmono]{droidmono}

\usepackage{amsmath,amssymb,amsthm,textcomp}
\usepackage{enumerate}
\usepackage{multicol}
\usepackage{float}
\usepackage{hyperref}
\usepackage{tikz}

\usepackage{geometry}
\geometry{left=25mm,right=25mm,%
bindingoffset=0mm, top=20mm,bottom=20mm}


\linespread{1.3}

\newcommand{\linia}{\rule{\linewidth}{0.5pt}}

% custom theorems if needed
\newtheoremstyle{mytheor}
    {1ex}{1ex}{\normalfont}{0pt}{\scshape}{.}{1ex}
    {{\thmname{#1 }}{\thmnumber{#2}}{\thmnote{ (#3)}}}

\theoremstyle{mytheor}
\newtheorem{defi}{Definition}

% my own titles
\makeatletter
\renewcommand{\maketitle}{
\begin{center}
\vspace{2ex}
{\huge \textsc{\@title}}
\vspace{1ex}
\\
\linia\\
\@author \hfill \@date
\vspace{4ex}
\end{center}
}
\makeatother
%%%

% custom footers and headers
% \usepackage{fancyhdr}
% \pagestyle{fancy}
% \lhead{}
% \chead{}
% \rhead{}
% \lfoot{Exercise 2.1}
% \cfoot{}
% \rfoot{Page \thepage}
% \renewcommand{\headrulewidth}{0pt}
% \renewcommand{\footrulewidth}{0pt}
\pagenumbering{gobble}

% code listing settings
% \usepackage{listings}
% \lstset{
%     language=Python,
%     basicstyle=\ttfamily\small,
%     aboveskip={1.0\baselineskip},
%     belowskip={1.0\baselineskip},
%     columns=fixed,
%     extendedchars=true,
%     breaklines=true,
%     tabsize=4,
%     prebreak=\raisebox{0ex}[0ex][0ex]{\ensuremath{\hookleftarrow}},
%     frame=lines,
%     showtabs=false,
%     showspaces=false,
%     showstringspaces=false,
%     keywordstyle=\color[rgb]{0.627,0.126,0.941},
%     commentstyle=\color[rgb]{0.133,0.545,0.133},
%     stringstyle=\color[rgb]{01,0,0},
%     numbers=left,
%     numberstyle=\small,
%     stepnumber=1,
%     numbersep=10pt,
%     captionpos=t,
%     escapeinside={\%*}{*)}
% }

\usepackage{minted}
\usemintedstyle{borland}
\definecolor{bg}{rgb}{0.95,0.95,0.95}

%%%----------%%%----------%%%----------%%%----------%%%

\begin{document}

\title{NumMet Exercise 3.2}

\author{Markus Tajakka}

\date{13/02/2026}

\maketitle

No AI has been used in these excersices.

\section*{Code}

\subsection*{main.py}
\inputminted[bgcolor=bg]{python}{main.py}

\subsection*{shallow\_water.py}
\inputminted[bgcolor=bg]{python}{../shallow_water.py}

The code can also be found at \url{https://github.com/MTajakka/NumMet/tree/master}

\newpage

\section*{Plots}
\begin{figure}[H]
    \centering
    \includegraphics[width=0.7\linewidth]{3_2.png}
    \caption{Hovmöeller diagram}
    \label{fig:d}
\end{figure}

\section*{Comments}

The wave crosses the boundary at around half of the first day. At atound 50000 s mark.
From the data, this peak is at 48100 s. During this time, the wave has moved from the 
middle to the boundary and then crossed the whole space once. Dividing this distance 
with the time, gives us speed oof 30.87 m/s which is quite fast. This is larger than 
the advection speed, by a large margin. From the prevoius excersice, the advection 
term is three orderds of magintude smaller than the term caused by the shape.
The Courant number is 0.31.

\end{document}
