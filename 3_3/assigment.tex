%%% Template originaly created by Karol Kozioł (mail@karol-koziol.net) and modified for ShareLaTeX use

\documentclass[a4paper,11pt]{article}

\usepackage[T1]{fontenc}
\usepackage[utf8]{inputenc}
\usepackage{graphicx}
\usepackage{xcolor}

\renewcommand\familydefault{\sfdefault}
\usepackage{tgheros}
% \usepackage[defaultmono]{droidmono}

\usepackage{amsmath,amssymb,amsthm,textcomp}
\usepackage{enumerate}
\usepackage{multicol}
\usepackage{float}
\usepackage{hyperref}
\usepackage{tikz}

\usepackage{geometry}
\geometry{left=25mm,right=25mm,%
bindingoffset=0mm, top=20mm,bottom=20mm}


\linespread{1.3}

\newcommand{\linia}{\rule{\linewidth}{0.5pt}}

% custom theorems if needed
\newtheoremstyle{mytheor}
    {1ex}{1ex}{\normalfont}{0pt}{\scshape}{.}{1ex}
    {{\thmname{#1 }}{\thmnumber{#2}}{\thmnote{ (#3)}}}

\theoremstyle{mytheor}
\newtheorem{defi}{Definition}

% my own titles
\makeatletter
\renewcommand{\maketitle}{
\begin{center}
\vspace{2ex}
{\huge \textsc{\@title}}
\vspace{1ex}
\\
\linia\\
\@author \hfill \@date
\vspace{4ex}
\end{center}
}
\makeatother
%%%

% custom footers and headers
% \usepackage{fancyhdr}
% \pagestyle{fancy}
% \lhead{}
% \chead{}
% \rhead{}
% \lfoot{Exercise 2.1}
% \cfoot{}
% \rfoot{Page \thepage}
% \renewcommand{\headrulewidth}{0pt}
% \renewcommand{\footrulewidth}{0pt}
\pagenumbering{gobble}

% code listing settings
% \usepackage{listings}
% \lstset{
%     language=Python,
%     basicstyle=\ttfamily\small,
%     aboveskip={1.0\baselineskip},
%     belowskip={1.0\baselineskip},
%     columns=fixed,
%     extendedchars=true,
%     breaklines=true,
%     tabsize=4,
%     prebreak=\raisebox{0ex}[0ex][0ex]{\ensuremath{\hookleftarrow}},
%     frame=lines,
%     showtabs=false,
%     showspaces=false,
%     showstringspaces=false,
%     keywordstyle=\color[rgb]{0.627,0.126,0.941},
%     commentstyle=\color[rgb]{0.133,0.545,0.133},
%     stringstyle=\color[rgb]{01,0,0},
%     numbers=left,
%     numberstyle=\small,
%     stepnumber=1,
%     numbersep=10pt,
%     captionpos=t,
%     escapeinside={\%*}{*)}
% }

\usepackage{minted}
\usemintedstyle{borland}
\definecolor{bg}{rgb}{0.95,0.95,0.95}

%%%----------%%%----------%%%----------%%%----------%%%

\begin{document}

\title{NumMet Exercise 3.3}

\author{Markus Tajakka}

\date{13/02/2026}

\maketitle

No AI has been used in these excersices.

\section*{Code}

\subsection*{main.py}
\inputminted[bgcolor=bg]{python}{main.py}

\subsection*{shallow\_water.py}
\inputminted[bgcolor=bg]{python}{../shallow_water.py}

The code can also be found at \url{https://github.com/MTajakka/NumMet/tree/master}

\newpage

\section*{Plots}

\begin{figure}[H]
    \centering
    \includegraphics[width=0.7\linewidth]{Hovmoller_diagram_120.png}
    \caption{Hovmöeller diagram at dt = 120}
    \label{fig:h}
\end{figure}

\begin{figure}[H]
    \centering
    \includegraphics[width=0.7\linewidth]{Hovmoller_diagram_320.png}
    \caption{Hovmöeller diagram, before breaking, at dt = 320}
    \label{fig:d}
\end{figure}

\section*{Comments}

The first failure point came from time step becoming too large at dt = 320 s. 
I didn't write a algorithm to find the specific value and this was more of trial 
and error. Due to the lenght of each run, it takes some time for them to finish and 
having hudreds of runs, takes a lot of time. So I found the limit by increasing the 
timestep by 100s until breaking, and the by 100 from the previous. 

As seen in the animation, the values in the model explode and causes the values to 
overflow. Luckily numpy can raise an error in this case so the model can fail saifly.

Plotting a Hovmöeller diagram for the model before it breaks \ref{fig:d}, shows it 
is similar to a stable model \ref{fig:h}. Meaning the have the same speed for waves 
of around 31 m/s. 
What is different is The Courant number. For the model that breaks, the number is 
Du = 1.00. This seems almost too much of a coincidence and the number is actually 
indicator for model stability.

\end{document}
